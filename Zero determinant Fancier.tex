\documentclass[12pt]{article}
\usepackage{euler}
\usepackage[margin=1in]{geometry}  % set the margins to 1in on all sides
\usepackage{graphicx}              % to include figures
\usepackage{amsmath}               % great math stuff
\usepackage{amsfonts}              % for blackboard bold, etc
\usepackage{amsthm,amssymb}
\usepackage[hidelinks]{hyperref}
\usepackage[framemethod=tikz]{mdframed}
\usetikzlibrary{calc}
\usepackage{chngcntr}
\usepackage{lipsum}


% various theorems, numbered by section

\newtheorem{thm}{Theorem}[section]
\newtheorem{lem}[thm]{Lemma}
\newtheorem{prop}[thm]{Proposition}
\newtheorem{cor}[thm]{Corollary}
\newtheorem{conj}[thm]{Conjecture}

\theoremstyle{definition}
\newtheorem{defn}[thm]{Definition}
\newtheorem{defns}[thm]{Definitions}
\newtheorem{con}[thm]{Construction}
\newtheorem{exmp}[thm]{Example}
\newtheorem{exmps}[thm]{Examples}
\newtheorem{notn}[thm]{Notation}
\newtheorem{notns}[thm]{Notations}
\newtheorem{addm}[thm]{Addendum}
\newtheorem{exer}[thm]{Exercise}

\theoremstyle{remark}
\newtheorem{rem}[thm]{Remark}
\newtheorem{rems}[thm]{Remarks}
\newtheorem{warn}[thm]{Warning}
\newtheorem{sch}[thm]{Scholium}
\DeclareMathOperator{\id}{id}
\usepackage{cleveref}	
\newcommand{\bd}[1]{\mathbf{#1}}  % for bolding symbols
\newcommand{\RR}{\mathbb{R}}      % for Real numbers
\newcommand{\ZZ}{\mathbb{Z}}      % for Integers
\newcommand{\col}[1]{\left[\begin{matrix} #1 \end{matrix} \right]}
\newcommand{\comb}[2]{\binom{#1^2 + #2^2}{#1+#2}}

%% counters
\newcounter{theorem}
\newcounter{lemma}
\newcounter{corollary}
\counterwithin{theorem}{section}
\counterwithin{lemma}{section}
\counterwithin{corollary}{section}
% names for the structures
\newcommand\theoname{Theorem}
\newcommand\lemmname{Lemma}
\newcommand\coroname{Corollary}

\makeatletter
% mdf key for the eventual notes in the structures
\def\mdf@@mynote{}
\define@key{mdf}{mynote}{\def\mdf@@mynote{#1}}

% style for theorems
\mdfdefinestyle{mytheo}{
settings={\refstepcounter{theorem}},
linewidth=1pt,
innertopmargin=1.5\baselineskip,
roundcorner=10pt,
backgroundcolor=blue!05,
linecolor=orange,
singleextra={
  \node[xshift=10pt,thick,draw=blue,fill=blue!20,rounded corners,anchor=west] at (P-|O) %
  {\strut{\bfseries\theoname~\thetheorem}\ifdefempty{\mdf@@mynote}{}{~(\mdf@@mynote)}};
},
firstextra={
  \node[xshift=10pt,thick,draw=blue,fill=blue!20,rounded corners,anchor=west] at (P-|O) %
  {\strut{\bfseries\theoname~\thetheorem}\ifdefempty{\mdf@@mynote}{}{~(\mdf@@mynote)}};
}
}
% style for coroll
\mdfdefinestyle{mycoro}{
settings={\refstepcounter{corollary}},
linewidth=1pt,
innertopmargin=1.5\baselineskip,
roundcorner=10pt,
backgroundcolor=green!05,
linecolor=orange,
singleextra={
  \node[xshift=10pt,thick,draw=yellow,fill=yellow!20,rounded corners,anchor=west] at (P-|O) %
  {\strut{\bfseries\coroname~\thecorollary}\ifdefempty{\mdf@@mynote}{}{~(\mdf@@mynote)}};
},
firstextra={
  \node[xshift=10pt,thick,draw=yellow,fill=yellow!20,rounded corners,anchor=west] at (P-|O) %
  {\strut{\bfseries\coroname~\thecorollary}\ifdefempty{\mdf@@mynote}{}{~(\mdf@@mynote)}};
}
}

% style for lemmas
\mdfdefinestyle{mylemm}{
settings={\refstepcounter{lemma}},
linewidth=1pt,
innertopmargin=1.5\baselineskip,
roundcorner=10pt,
backgroundcolor=red!05,
linecolor=red!70!black,
singleextra={
  \node[xshift=10pt,thick,draw=green,fill=green!20,rounded corners,anchor=west] at (P-|O) %
  {\strut{\bfseries\lemmname~\thelemma}\ifdefempty{\mdf@@mynote}{}{~(\mdf@@mynote)}};
},
firstextra={
  \node[xshift=10pt,thick,draw=green,fill=green!20,rounded corners,anchor=west] at (P-|O) %
 {\strut{\bfseries\lemmname~\thelemma}\ifdefempty{\mdf@@mynote}{}{~(\mdf@@mynote)}};
}
}
% some auxiliary environments
\newmdenv[style=mytheo]{theor}
\newmdenv[style=mylemm]{lemm}
\newmdenv[style=mycoro]{coro}
% the actual environments
\newenvironment{theorem}[1][]
  {\begin{theor}[mynote=#1]}
  {\end{theor}}
\newenvironment{lemma}[1][]
  {\begin{lemm}[mynote=#1]}
  {\end{lemm}}
\newenvironment{corollary}[1][]
  {\begin{coro}[mynote=#1]}
  {\end{coro}}
\makeatother

\begin{document}


\nocite{}

\title{Zero Determinants}
\date{October 31, 2014}
\author{Miliyon T.}
\maketitle

\begin{abstract}
  The determinant of a matrix has a versatile application. For instance if the determinant of a coefficient matrix of a given system is zero, then it follows that the equations(which form the system) are not linearly independent. The determinant is also helpful in determining whether a given matrix is invertible or not. If the matrix contain zero row(column) or if one row(column) of the matrix is a multiple of the other, then the determinant of that matrix becomes zero. A matrix with zero determinant is known as a \textbf{singular matrix} and it is known that singular matrices are not invertible. In this paper, we will try to study some simple property which characterize singular matrices.
\end{abstract}

\section{Definitions}

\begin{itemize}
\item \textbf{Matrix}: is a rectangular array of numbers, symbols, or expressions,
arranged in rows and columns.\\
e.g. Matrix $A$ of order $n$ is given by
  \begin{gather*}
  A = \begin{bmatrix}
      a_{11}  & \cdots & a_{1n} \\
      \vdots  & \ddots & \vdots \\
      a_{n1}  & \cdots & a_{nn}
    \end{bmatrix}
  \end{gather*}

\item \textbf{Square Matrix}: is a matrix with equal number of rows and columns.

\item \textbf{Determinant}: is a useful value that can be computed from a square matrix.

\textmd{\textsc{Determinant for\footnote{For $3\times3$ matrix one can use \textit{Sarrus' rule} to determine the determinant.} $2\times 2$ matrix}}:
Given a matrix
\[
  A = \begin{bmatrix}
      a   & b \\
       c  & d
    \end{bmatrix}
\]
\[
\det(A)=ad-bc.
\]
\item \textbf{Minor and Cofactor:} If $A$ is a square matrix, then the \textbf{minor} $A_{ij}$ of the element $a_{ij}$ is the determinant of the matrix obtained by deleting the $i^{th}$ row and $j^{th}$ column of $A$. Then the cofactor $\Delta_{ij}$ is given by $\Delta_{ij} =(-1)^{i+j}  \det(A_{ij})$.
\end{itemize}

\begin{itemize}
\item \textbf{Laplace Expansion(for\footnote{Leibniz formula can be also used to calculate the determinant of $n\times n$ matrix, $$\det(A)=\sum_{\sigma\in S_n}\text{sgn}(\sigma)\prod_{i=1}^{n}a_{\sigma(i),i}$$} $n\times n$ matrix)}\\
Let $A=(a_{ij})$ be a square matrix of order $n$ where $n\geq 3.$ Then, $\det(A)$ can be expressed as a cofactor expansion by using any row of $A$.
\begin{equation}
\det(A)= \sum_{j=1}^n (-1)^{i+j} a_{ij} \det(A_{ij})
\end{equation}
\end{itemize}
\section{Facts}

\begin{lemma}\label{lem}
The determinant of any $3\times 3$ matrix with \textbf{row(column) of common difference} is zero.
\end{lemma}
\begin{proof}
Let's take a $3\times3$ matrix
\[
  A = \begin{bmatrix}
      a  & b & c \\
      d  & e & f \\
      g  & h & i
    \end{bmatrix}
\]
Since the matrix is with \textbf{row of common difference}\footnote{In each row the entries are differ by a constant. This constant may \textbf{vary} from one row to the other.} the rows satisfy some arithmetic progression. Say,
\begin{align*}
b=a+x, &\qquad c=b+x\\
e=d+y, &\qquad f=e+y\\
h=g+z, &\qquad i=h+z
\end{align*}
Then the matrix become
\begin{align*}
  A =\begin{bmatrix}
      a  & a+x & b+x \\
      d  & d+y & e+y \\
      g  & g+z & h+z
    \end{bmatrix}
    =\begin{bmatrix}
      a  & a+x & a+2x \\
      d  & d+y & d+2y \\
      g  & g+z & g+2z
    \end{bmatrix}
\end{align*}
Computing the determinant
\begin{align*}
  det(A) &=\begin{vmatrix}
      a  & a+x & a+2x \\
      d  & d+y & d+2y \\
      g  & g+z & g+2z
    \end{vmatrix}
    =a\cdot\begin{vmatrix}
    d+y  & d+2y \\
     g+z  & g+2z
    \end{vmatrix}
    -d\cdot\begin{vmatrix}
    a+x  & a+2x \\g+z  & g+2z
    \end{vmatrix}
    +g\cdot\begin{vmatrix}
     a+x  & a+2x \\
     d+y  & d+2y
     \end{vmatrix}\\
&= a\cdot(dz-gy)-d\cdot(az-gx)+g\cdot(ay-dx)\\
&=adz-agy-a dz+dg x+a gy-d gx\\
&=(ad z-a dz)+(a gy-ag y)+(dg x-d gx)=0
\end{align*}
\end{proof}

\begin{theorem}[Zero determinant Theorem]
Given any square matrix $A$ of order $n\ge3$. If $A$ is a matrix with \textbf{row(column) of common difference}, then the determinant of $A$ is zero.
\end{theorem}

\begin{proof}
Consider an $n\times n$ matrix $A$ with row of common difference
\[
  A =\begin{bmatrix}
      a_{11}  & \cdots & a_{1n} \\
      \vdots  & \ddots & \vdots \\
      a_{n1}  & \cdots & a_{nn}
    \end{bmatrix}
\]
From Laplace expansion we have
$$
 \det(A)= \sum_{j=1}^n (-1)^{i+j} a_{ij} \det(A_{ij})
$$

$$
=(-1)^{1+1} a_{11} \det(A_{11})+(-1)^{1+2} a_{12} \det(A_{12})+\cdot\cdot\cdot+(-1)^{n+1} a_{1n} \det(A_{1n})
$$
If each matrices $A_{11},A_{12},\cdot\cdot\cdot,A_{1n}$ are $3\times3$ matrix then we are done. But if not we will repeat this process until we get a $3\times3$ matrix.\\
Then this $3\times3$ matrix is a matrix with \textbf{row of common difference} since the bigger matrix $A$(from which it constructed) itself is a matrix with \textbf{row of common difference}.\\
Thus, the determinant of each matrix $A_{11},A_{12},\cdot\cdot\cdot,A_{1n}$ is zero by lemma (\ref{lem}). Then
\begin{align*}
\det(A)&=(-1)^{1+1} a_{11} \det(A_{11})+(-1)^{1+2} a_{12} \det(A_{12})+\cdot\cdot\cdot+(-1)^{n+1} a_{1n} \det(A_{1n})\\
&=(-1)^{1+1} a_{11} (0)+(-1)^{1+2} a_{12} (0)+\cdot\cdot\cdot+(-1)^{n+1} a_{1n} (0)\\
&=0
\end{align*}
Therefore, $\det(A)=0$.
\end{proof}

\begin{coro}
If $A$ is a square matrix with consecutive entries, then the determinant of $A$ is zero.
\end{coro}

\subsection*{Conclusion}
This is just a nice way of saying; if a matrix is with row(column) of common difference, then somehow we can express one row(column) of that matrix as a linear combination of the other.\\
A trivial matrix with zero determinant is; a matrix with zero row(column) or a matrix containing a row(column) which is a scalar multiple of some other row(column). But by using our result, we can easily generate a non trivial matrix of determinant zero.

\newpage

 \begin{thebibliography}{9}

\bibitem{amsshort}
[Serg Lang]
Linear Algebra,
Addison-Wesley Publishing, 1972.

\bibitem{notsoshort}
[Kolman \S ~Hill]
Introduction to Linear Algebra with Applications, 2000.

\bibitem{May}
[Demissu Gemeda]
An Introduction to Linear Algebra,
AAU Press, 2000.

\end{thebibliography}

\end{document}
