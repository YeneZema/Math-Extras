\documentclass[12pt]{article}

\usepackage[margin=1in]{geometry}  % set the margins to 1in on all sides
\usepackage{graphicx}              % to include figures
\usepackage{amsmath}               % great math stuff
\usepackage{amsfonts}              % for blackboard bold, etc
\usepackage{amsthm}                % better theorem environments
\usepackage{amsthm,amssymb}
\newtheorem{thm}{Theorem}[section]
\newtheorem{lem}[thm]{Lemma}
\newtheorem{prop}[thm]{Proposition}
\newtheorem{cor}[thm]{Corollary}
\newtheorem{conj}[thm]{Conjecture}

\DeclareMathOperator{\id}{id}

\newcommand{\bd}[1]{\mathbf{#1}}  % for bolding symbols
\newcommand{\RR}{\mathbb{R}}      % for Real numbers
\newcommand{\ZZ}{\mathbb{Z}}      % for Integers
\newcommand{\col}[1]{\left[\begin{matrix} #1 \end{matrix} \right]}
\newcommand{\comb}[2]{\binom{#1^2 + #2^2}{#1+#2}}

\begin{document}


\nocite{}

\title{\textbf{Euler Formula}}

\author{Miliyon T.}
\maketitle


\section{Euler factorial formula}

\begin{thm}[Euler formula]
Let  a and n be nonnegative integers with $a\geq n$. Then
$$ n!=\sum_{k=0}^n (-1)^k\binom{n}{k}(a-k)^n $$

Which is equivalent to

$$
n!=a^n - \binom{n}{1}(a-1)^n + \binom{n}{2}(a-2)^n - \binom{n}{3}(a-3)^n + \cdot\cdot\cdot + (-1)^n\binom{n}{n}(a-n)^n $$

\begin{proof}
It is trivial For $n=1$ 
\begin{align*}
1!=a^1-\binom{1}{1}(a-1)^1 \\
1=a-(a-1)=1
\end{align*}
Then, we assume it is true for n
$$ n!=\sum_{k=0}^n (-1)^k\binom{n}{k}(a-k)^n $$
Now let's proof that it is true for $n+1$
\begin{align*}
\sum_{k=0}^{n+1} (-1)^k\binom{n+1}{k}(a-k)^{n+1}&=\sum_{k=0}^n (-1)^k\binom{n}{k}(a-k)^n \\
&=\sum_{k=0}^n (-1)^k\binom{n}{k}(a-k)^n
\end{align*}


\end{proof}
\end{thm} 


 \begin{thebibliography}{9}

\bibitem{May}
[Demissu Gemeda]
Topics in Linear Algebra,
Addis Ababa University. 2005

\bibitem{amsshort}
[Sheldon Axler]
Linear Algebra Done Right,
Springer Publishing. 1997

\bibitem{notsoshort}
[Caroline Laroche Turnage]
Selected Proofs of Fermat's Little Theorem  and Wilson's Theorem.

\end{thebibliography}

\end{document}
