\documentclass[12pt]{article}

\usepackage[margin=1in]{geometry}  % set the margins to 1in on all sides
\usepackage{graphicx}              % to include figures
\usepackage{amsmath}               % great math stuff
\usepackage{amsfonts}              % for blackboard bold, etc
\usepackage{amsthm}                % better theorem environments
\usepackage{amsthm,amssymb}
\usepackage{graphicx,wrapfig,lipsum}
\usepackage{amsthm}                % better theorem environments

\usepackage{xcolor,cancel}
\newcommand\hcancel[2][black]{\setbox0=\hbox{$#2$}%
\rlap{\raisebox{.45\ht0}{\textcolor{#1}{\rule{\wd0}{1pt}}}}#2}


\begin{document}

\nocite{}

\title{Epsilon-Delta Proof}

\author{Miliyon T.}
\maketitle
\textcolor[rgb]{1.00,0.00,0.00}{\textbf{{\Large Single Variable}}}\\

\begin{wrapfigure}{l}{0.41\textwidth}
		\includegraphics[width=0.42\textwidth]{delta.png}
		\\	% this spacer is needed to make the text on the right fit OK
	\end{wrapfigure}

\textbf{\large Precise Definition of Limit}\\
Let $f(x)$ be defined on an open interval about $x_0$, except possibly at $x_0$ itself. We say that the limit of $f(x)$ as $x$ approaches $x_0$ is the number L, written as
$$
\lim_{x\rightarrow x_0} f(x)=L
$$
For every number $\epsilon >0$ , there exists a corresponding number $\delta >0$. Such that for all $x$.
$$0<|x-x_0|<\delta\Rightarrow|f(x)-L|<\epsilon$$
\\
\\
\\

\textcolor[rgb]{0.50,0.50,0.00}{\textbf{{\Large Historical Note }}}
\\
\begin{wrapfigure}{l}{0.3\textwidth}
		\includegraphics[width=0.3\textwidth]{1.jpg}
\caption{Cauchy}\label{fig:1}	
\end{wrapfigure}

The foundations of the rigorous study of \emph{analysis}
were laid in the nineteenth century, notably by the
mathematicians Cauchy and Weierstrass. \\
\textcolor[rgb]{0.00,0.00,1}{\textbf{Augustine Louis Cauchy}} \\
The foundations of the rigorous study of \emph{analysis}
were laid in the nineteenth century, notably by the
mathematicians Cauchy and Weierstrass. Central to the
study of this subject are the formal definitions of
\emph{limits} and \emph{continuity}.
The foundations of the rigorous study of \emph{analysis}
were laid in the nineteenth century, notably by the
mathematicians Cauchy and Weierstrass.
\\
\\
\\


\textcolor[rgb]{0.00,0.00,1}{\textbf{Carl Weierstrass}}
\\
\begin{wrapfigure}{l}{0.3\textwidth}
		\includegraphics[width=0.3\textwidth]{2.jpg}
		\caption{Weierstrass}\label{fig:1}	
\end{wrapfigure}

The foundations of the rigorous study of \emph{analysis}
were laid in the nineteenth century, notably by the
mathematicians Cauchy and Weierstrass. Central to the
study of this subject are the formal definitions of
\emph{limits} and \emph{continuity}.
The foundations of the rigorous study of \emph{analysis}
were laid in the nineteenth century, notably by the
mathematicians Cauchy and Weierstrass. Central to the
study of this subject are the formal definitions of
\emph{limits} and \emph{continuity}.

Let $D$ be a subset of $\bf R$ and let
$f \colon D \to \mathbf{R}$ be a real-valued function on
$D$. The function $f$ is said to be \emph{continuous} on
$D$ if, for all $\epsilon > 0$ and for all $x \in D$,
there exists some $\delta > 0$ (which may depend on $x$)
such that if $y \in D$ satisfies

One may readily verify that if $f$ and $g$ are continuous
functions on $D$ then the functions $f+g$, $f-g$ and
$f.g$ are continuous. If in addition $g$ is everywhere
non-zero then $f/g$ is continuous.
\\
\\
The foundations of the rigorous study of \emph{analysis}
were laid in the nineteenth century, notably by the
mathematicians Cauchy and Weierstrass. Central to the
study of this subject are the formal definitions of
\emph{limits} and \emph{continuity}.
The foundations of the rigorous study of \emph{analysis}
were laid in the nineteenth century, notably by the
mathematicians Cauchy and Weierstrass. Central to the
study of this subject are the formal definitions of
\emph{limits} and \emph{continuity}.

Let $D$ be a subset of $\bf R$ and let
$f \colon D \to \mathbf{R}$ be a real-valued function on
$D$. The function $f$ is said to be \emph{continuous} on
$D$ if, for all $\epsilon > 0$ and for all $x \in D$,
there exists some $\delta > 0$ (which may depend on $x$)
such that if $y \in D$ satisfies

One may readily verify that if $f$ and $g$ are continuous
functions on $D$ then the functions $f+g$, $f-g$ and
$f.g$ are continuous. If in addition $g$ is everywhere
non-zero then $f/g$ is continuous.

\newpage

\textcolor[rgb]{1.00,0.00,0.00}{\textbf{{\Large  Multi-variable}}}\\

    \textbf{\large Precise Definition}\\
Let $f(x)$ be defined on an open interval about $x_0$, except possibly at $x_0$ itself. We say that the limit of $f(x)$ as $x$ approaches $x_0$ is the number L, written as
$$
\lim_{(x,y)\rightarrow (x_0,y_0)} f(x,y)=L
$$
If ,for every number $\epsilon >0$ , there exists a corresponding number $\delta >0$. Such that for all $x$.
$$0<|x-x_0|<\delta\Rightarrow|f(x)-L|<\epsilon$$

\begin{wrapfigure}{l}{0.41\textwidth}
		\includegraphics[width=0.42\textwidth]{delta.png}
		\\	% this spacer is needed to make the text on the right fit OK
	\end{wrapfigure}

The foundations of the rigorous study of \emph{analysis}
were laid in the nineteenth century, notably by the
mathematicians Cauchy and Weierstrass. Central to the
study of this subject are the formal definitions of
\emph{limits} and \emph{continuity}.

Let $D$ be a subset of $\bf R$ and let
$f \colon D \to \mathbf{R}$ be a real-valued function on
$D$. The function $f$ is said to be \emph{continuous} on
$D$ if, for all $\epsilon > 0$ and for all $x \in D$,
there exists some $\delta > 0$ (which may depend on $x$)
such that if $y \in D$ satisfies
\[ |y - x| < \delta \]
then
\[ |f(y) - f(x)| < \epsilon. \]

One may readily verify that if $f$ and $g$ are continuous
functions on $D$ then the functions $f+g$, $f-g$ and
$f.g$ are continuous. If in addition $g$ is everywhere
non-zero then $f/g$ is continuous.
 \begin{thebibliography}{9}

\bibitem{amsshort}
[Serg Lang]
Linear Algebra,
Addison-Wesley Publishing. 1972

\bibitem{notsoshort}
[Kolman \S  Hill]
Introduction to Linear Algebra with Applications.2000

\bibitem{May}
[Demissu Gemeda]
An Introduction to Linear Algebra,
Addis Ababa University Press.2000

\end{thebibliography}

\end{document}
