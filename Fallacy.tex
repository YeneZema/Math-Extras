\documentclass[paper=a4, fontsize=12pt]{scrartcl} % A4 paper and 11pt font size

\usepackage[T1]{fontenc} % Use 8-bit encoding that has 256 glyphs
\usepackage{fourier} % Use the Adobe Utopia font for the document - comment this line to return to the LaTeX default
\usepackage[english]{babel} % English language/hyphenation
\usepackage{amsmath,amsfonts,amsthm} % Math packages
\usepackage{graphicx,wrapfig,lipsum}
\usepackage{fancyhdr} % Custom headers and footers
\pagestyle{fancyplain} % Makes all pages in the document conform to the custom headers and footers
\usepackage[margin=1in]{geometry}  % set the margins to 1in on all sides
\usepackage[hidelinks]{hyperref}
\usepackage{cleveref}

\newtheorem{thm}{Theorem}[section]
\newtheorem{lem}[thm]{Lemma}
\newtheorem{prop}[thm]{Proposition}
\newtheorem{cor}[thm]{Corollary}
\newtheorem{conj}[thm]{Conjecture}
\usepackage{sectsty} % Allows customizing section commands
\allsectionsfont{\centering \normalfont\scshape} % Make all sections centered, the default font and small caps



\fancyhead{} % No page header - if you want one, create it in the same way as the footers below
\fancyfoot[L]{} % Empty left footer
\fancyfoot[C]{} % Empty center footer
\fancyfoot[R]{\thepage} % Page numbering for right footer
\renewcommand{\headrulewidth}{0pt} % Remove header underlines
\renewcommand{\footrulewidth}{0pt} % Remove footer underlines
\setlength{\headheight}{13.6pt} % Customize the height of the header

% various theorems, numbered by section

\theoremstyle{definition}
\newtheorem{defn}[thm]{Definition}
\newtheorem{defns}[thm]{Definitions}
\newtheorem{con}[thm]{Construction}
\newtheorem{exmp}[thm]{Example}
\newtheorem{exmps}[thm]{Examples}
\newtheorem{notn}[thm]{Notation}
\newtheorem{notns}[thm]{Notations}
\newtheorem{addm}[thm]{Addendum}
\newtheorem{exer}[thm]{Exercise}

\theoremstyle{remark}
\newtheorem{rem}[thm]{Remark}
\newtheorem{rems}[thm]{Remarks}
\newtheorem{warn}[thm]{Warning}
\newtheorem{sch}[thm]{Scholium}
\DeclareMathOperator{\id}{id}

\newcommand{\bd}[1]{\mathbf{#1}}  % for bolding symbols
\newcommand{\RR}{\mathbb{R}}      % for Real numbers
\newcommand{\ZZ}{\mathbb{Z}}      % for Integers
\newcommand{\col}[1]{\left[\begin{matrix} #1 \end{matrix} \right]}
\newcommand{\comb}[2]{\binom{#1^2 + #2^2}{#1+#2}}
%----------------------------------------------------------------------------------------
%	TITLE SECTION
%----------------------------------------------------------------------------------------

\newcommand{\horrule}[1]{\rule{\linewidth}{#1}} % Create horizontal rule command with 1 argument of height

\title{	
\normalfont \normalsize
\textsc{Addis Ababa university} \\ [25pt] % Your university, school and/or department name(s)
\horrule{0.5pt} \\[0.4cm] % Thin top horizontal rule
\huge Mathematical Fallacies and Paradoxes \\ % The assignment title
\horrule{2pt} \\[0.5cm] % Thick bottom horizontal rule
}

\author{Miliyon T.} % Your name

\date{\normalsize\today} % Today's date or a custom date

\begin{document}

\maketitle % Print the title
\begin{abstract}{Abstract: }
  \textbf{Mathematical fallacies} are errors, typically committed with an intent to deceive, that occur in a mathematical proof or argument. A fallacy in an argument doesn't necessarily mean that the conclusion is necessarily incorrect, only that the argument itself is wrong. However, fallacious arguments can have surprising conclusions. Apart from a mathematical fallacy a \textbf{paradox} is a statement that goes against our intuition but may be true, or a statement that is or appears to be self-contradictory. Mathematical paradoxes result from either counter-intuitive properties of infinity, or self-reference.
\end{abstract}

\subsection*{\textbf{First Fallacy}}
A fallacy due to John Bernoulli\index{Bernoulli, John}, may be stated as follows.
We have  $(-1)^2 = 1\,.  $
Take logarithms,  $2 \log(-1) = \log 1 = 0\,.$
\begin{align*}
   \log(-1) &= 0\, \\
         -1 &= e^0\,\\
         -1 &= 1\,\\
\end{align*}

The same argument may be expressed thus. Let $x$ be a quantity which satisfies the equation
$e^x = -1$\\
Square both sides
  $$ e^{2x} = 1$$
  $$\Rightarrow 2x = 0$$
  $$\Rightarrow x = 0$$
  $$\Rightarrow e^x = e^0$$
But $e^x = -1$ and $e^0 = 1$,\\
$$  -1 = 1$$

\subsection*{\textbf{Second Fallacy}}
Suppose that $a = b$, then
\begin{align*}
ab & =  a^2\, \\
  ab-b^2 & =  a^2 - b^2\, \\
 b(a-b) & =  (a+b)(a-b)\, \\
 b & =  a + b\, \\
  b & =  2b\,\\
 1 & =  2\,
\end{align*}

\subsection*{\textbf{Third Fallacy}}
Let $a$ and $b$ be two unequal numbers, and let $c$ be their arithmetic mean, hence
\begin{align*}
          a + b &= 2c\,\\
 (a + b)(a - b) &= 2c(a - b)\,\\
      a^2 - 2ac &= b^2 - 2bc\,\\
a^2 - 2ac + c^2 &= b^2 - 2bc + c^2\,\\
      (a - c)^2 &= (b - c)^2\,\\
              a &= b\,
\end{align*}



\subsection*{\textbf{Fourth Fallacy}}
From Taylor's expansion, we know that
\[
  \log(1 + x) = x - \tfrac{1}{2}x^2 + \tfrac{1}{3}x^3 - \dotsb\,.
\]
If $x = 1$, the resulting series is convergent; hence we have
\begin{align*}
  \log 2 &= 1 - \tfrac{1}{2} + \tfrac{1}{3} - \tfrac{1}{4}
+ \tfrac{1}{5} - \tfrac{1}{6} + \tfrac{1}{7} - \tfrac{1}{8}
+ \tfrac{1}{9} - \dotsb\,.  \\
 2 \log 2 &= 2 - 1 + \tfrac{2}{3} - \tfrac{1}{2}
+ \tfrac{2}{5} - \tfrac{1}{3} + \tfrac{2}{7} - \tfrac{1}{4}
+ \tfrac{2}{9} - \dotsb\,.
\end{align*}
Taking those terms together which have a common denominator,
we obtain
\begin{align*}
  2 \log 2 & =  1 + \frac{1}{3} - \frac{1}{2} + \frac{1}{5} +
  \frac{1}{7} - \frac{1}{4} + \frac{1}{9}-\dotsb \\ % inserted final -
           & =  1 - \frac{1}{2} + \frac{1}{3} - \frac{1}{4} +
  \frac{1}{5}-\dotsb \\
            & =  \log 2\, \\
          2 &= 1\,\\
\end{align*}

\subsection*{\textbf{Fifth Fallacy}} This fallacy is very similar to that last
given. We have
\begin{align*}
  \log 2 & = \textstyle 1 - \frac{1}{2} + \frac{1}{3} - \frac{1}{4} +
  \frac{1}{5} - \frac{1}{6}+\dotsb \\
         & = \textstyle \left( 1 + \frac{1}{3} + \frac{1}{5} + \dotsb\right) -
  \left( \frac{1}{2} + \frac{1}{4}  + \frac{1}{6}+\dotsb\right) \\
         & = \textstyle \left \{  \left( 1 + \frac{1}{3} + \frac{1}{5}  +
  \dotsb \right)
  + \left( \frac{1}{2} + \frac{1}{4} + \frac{1}{6} + \dotsb\right)
  \right\} - 2 \left( \frac{1}{2}  + \frac{1}{4} + \frac{1}{6} +
  \dotsb \right) \\
         & = \textstyle \left\{ 1 + \frac{1}{2} + \frac{1}{3} + \dotsb\right\}
  - \left( 1 + \frac{1}{2} + \frac{1}{3} + \dotsb\right) \\
         & =  0\,
\end{align*}

The error in each of the foregoing examples is obvious, but
the fallacies in the next examples are concealed somewhat
better.

\subsection*{\textbf{Sixth Fallacy}} We can write the identity $\sqrt{-1} = \sqrt{-1}$
in the form
%
\begin{align*}
  \sqrt{\frac{-1}{1}} & = \sqrt{\frac{1}{-1}}\,  \\
   \frac{\sqrt{-1}}{\sqrt{1}} & =  \frac{\sqrt{1}}{\sqrt{-1}} \, \\
 (\sqrt{-1})^2 & = (\sqrt{1})^2 \,  \\
 -1 & = 1 \,
\end{align*}

\subsection*{\textbf{Seventh Fallacy}} Again, we have
\begin{align*}
 \sqrt{a} \cdot \sqrt{b} & = \sqrt{ab} \,  \\
 \sqrt{-1} \cdot \sqrt{-1} & = \sqrt{(-1) (-1)} \,  \\
 (\sqrt{-1})^2 & =\sqrt{1} \,  \\
 -1 & = 1 \,
\end{align*}

\subsection*{\textbf{Eighth Fallacy}}  The following demonstration depends on
the fact that an algebraical identity is true whatever be the
symbols used in it, and it will appeal only to those who are
familiar with this fact.

We have, as an identity,
\begin{align}\label{i}
  \sqrt{x-y} = i \sqrt{y-x}
\end{align}
where $i$ stands either for $+ \sqrt{-1}$ or for $- \sqrt{-1}$. Now an
\emph{identity} in $x$ and $y$ is necessarily true whatever numbers $x$
and $y$ may represent. First put $x = a$ and $y = b$,
\begin{align}\label{ii}
    \sqrt{a - b} = i \sqrt{b - a}
\end{align}
 Next put $x = b$ and $y = a$,
\begin{align}\label{iii}
    \sqrt{b - a} = i \sqrt{a - b}
\end{align}
Also since (\ref{i}) is an identity, it follows that in
(\ref{ii}) and (\ref{iii}) the symbol $i$ must be the same, that
is, it represents $+ \sqrt{-1}$ or $- \sqrt{-1}$ in both cases. Hence,
from (\ref{ii}) and (\ref{iii}), we have
\begin{align*}
 \sqrt{a-b}\; \sqrt{b - a} & =  i^2 \sqrt{b-a}\; \sqrt{a - b}\,, \\
    1 & =  i^2\,, \\
 1 &= -1 \, .
\end{align*}

%\subsection*{Ninth Fallacy} The following fallacy is due to
%D'Alembert
%{\textit{Opuscules mathématiques}, Paris, 1761, vol.~\textsc{i},
%p.~201.}. We know that if the product of two numbers is equal to the

%product of two other numbers, the numbers will be in proportion, and
%from the definition of a proportion it follows that if the first term
%is greater than the second, then the third term will be greater than
%the fourth: thus, if $ad=bc$, then $a:b = c:d$, and if in this
%proportion $a > b$, then $c > d$. Now if we put $a = d =1$ and $b = c
%= -1$ we have four numbers which satisfy the relation $ad = bc$ and such
%that $a>b$; hence, by the proposition, $c > d$, that is, $-1 > 1$, which is
%absurd.

%\subsection*{Tenth Fallacy} The mathematical theory of probability\index
%{Probabilities, Fallacies in} leads
%to various paradoxes: of these one specimen will suffice.
%Suppose three coins to be thrown up and the fact whether each
%comes down head or tail to be noticed. The probability that
%all three coins come down head is clearly $(\frac{1}{2})^3$, that is,
%is $\frac{1}{8}$;
%similarly the probability that all three come down tail is $\frac{1}{8}$:
%hence the probability that all the coins come down alike
%(i.e. either all of them heads or all of them tails) is $\frac{1}{4}$. But,
%of three coins thus thrown up, at least two must come down alike;
%now the probability that the third coin comes down head is $\frac{1}{2}$
%and the probability that it comes down tail is $\frac{1}{2}$, thus the
%probability that it comes down the same as the other two coins
%is $\frac{1}{2}$: hence the probability that all the coins come down alike
%is $\frac{1}{2}$. I leave to my readers to say whether either of these
%conflicting conclusions is right and if so, which\index
%{Arithmetical Fallacies|)}%
%\index{FallaciesArith@\textsc{Fallacies, Arithmetical}|)}.

\subsection*{\textbf{Ninth Fallacy}}
This paradox is due to my teacher Elias\index{DAlembert@D'Alembert}\footnote{Elias Bogale is one of a few teacher which I have been taught by.} \\
Consider the following limit\\
Case I
\begin{align*}
 \lim_{n\rightarrow \infty}\frac{1+2+\cdot\cdot\cdot+n}{n^2} & = \lim_{n\rightarrow \infty}\frac{1}{n^2}+ \lim_{n\rightarrow \infty}\frac{2}{n^2}+\cdot\cdot\cdot+\lim_{n\rightarrow \infty}\frac{n}{n^2}\, \\
                  & = \lim_{n\rightarrow \infty}\frac{1}{n^2}+ \lim_{n\rightarrow \infty}\frac{2}{n^2}+\cdot\cdot\cdot+\lim_{n\rightarrow \infty}\frac{1}{n}\, \\
               & =0+ 0+\cdot\cdot\cdot+0\, \\
               & = 0 \,
\end{align*}

Case II
\begin{align*}
 \lim_{n\rightarrow \infty}\frac{1+2+\cdot\cdot\cdot+n}{n^2} & = \lim_{n\rightarrow \infty}\frac{n(n+1)}{2n^2}\,\qquad (Gauss~Sum) \\
                  & = \lim_{n\rightarrow \infty}\frac{n^2+n}{2n^2}=\lim_{n\rightarrow \infty}\biggl[\frac{n^2}{2n^2}+\frac{n}{2n^2}\biggl] \, \\
               & =\lim_{n\rightarrow \infty} \frac{n^2}{2n^2}+\lim_{n\rightarrow \infty}\frac{n}{2n^2} \, \\
               & = \frac{1}{2}+0 \, \\
               & = \frac{1}{2} \,
\end{align*}
Hence from Case I and Case II we can conclude that
$$0=\frac{1}{2}$$
\newpage
\begin{thebibliography}{9}

\bibitem{May}
[Clifford A. Pickover] ~
A Passion for
Mathematics, 2005.

\bibitem{amsshort}
[E. A. Maxwell] ~
Fallacies in mathematics, 1963.

\bibitem{Jun}
[Bryan Bunch]~
Mathematical fallacies and paradoxes, 1982.

\bibitem{notsoshort}
[Wiki]~
\href{http://www.wikipedia.org}{Mathematical fallacy}.

\end{thebibliography}
\end{document}
