\documentclass[12pt]{article}

\usepackage[margin=1in]{geometry}  % set the margins to 1in on all sides
\usepackage{graphicx}              % to include figures
\usepackage{amsmath}               % great math stuff
\usepackage{amsfonts}              % for blackboard bold, etc
\usepackage{amsthm}                % better theorem environments
\usepackage{amsthm,amssymb}
\usepackage{xcolor,cancel}
\newcommand\hcancel[2][black]{\setbox0=\hbox{$#2$}%
\rlap{\raisebox{.45\ht0}{\textcolor{#1}{\rule{\wd0}{1pt}}}}#2}

% various theorems, numbered by section

\newtheorem{thm}{Theorem}[section]
\newtheorem{lem}[thm]{Lemma}
\newtheorem{prop}[thm]{Proposition}
\newtheorem{cor}[thm]{Corollary}
\newtheorem{conj}[thm]{Conjecture}

\theoremstyle{definition}
\newtheorem{defn}[thm]{Definition}
\newtheorem{defns}[thm]{Definitions}
\newtheorem{con}[thm]{Construction}
\newtheorem{exmp}[thm]{Example}
\newtheorem{notn}[thm]{Notation}
\newtheorem{notns}[thm]{Notations}
\newtheorem{addm}[thm]{Addendum}
\newtheorem{exer}[thm]{Exercise}
\newtheorem{rem}[thm]{Remark}
\theoremstyle{plain}


\theoremstyle{remark}
\newtheorem{rems}[thm]{Remarks}
\newtheorem{warn}[thm]{Warning}
\newtheorem{sch}[thm]{Scholium}
\DeclareMathOperator{\id}{id}

\newcommand{\bd}[1]{\mathbf{#1}}  % for bolding symbols
\newcommand{\RR}{\mathbb{R}}      % for Real numbers
\newcommand{\ZZ}{\mathbb{Z}}      % for Integers
\newcommand{\col}[1]{\left[\begin{matrix} #1 \end{matrix} \right]}
\newcommand{\comb}[2]{\binom{#1^2 + #2^2}{#1+#2}}




\begin{document}

\title{Proofs in Number Theory}
\author{Miliyon T.\\
Addis Ababa University \\
Ethiopia}
\date{February 24, 2014}
\maketitle

\begin{abstract}
  Number theory is one of the most elegant, abstract and the more beautiful branches of Mathematics. The Greatest mathematician Carl Friedreich Gauss once said that Mathematics is a Queen of Science and Theory of Number is the Queen of Mathematics. Although, Number Theory have been considered as non-applicable subject nowadays it is become crucial for Internet Cryptography. Here we scribe some elementary proofs in number theory.
\end{abstract}

\section{Definitions}

\begin{defn}
A nonempty set $S$ of real numbers is said to be well-ordered if every nonempty subset of $S$ has a least element.
\end{defn}

\begin{rem}
Every nonempty finite set of real numbers is well-ordered.
\end{rem}

\begin{defn}[\textbf{The Well-Ordering Principle}]
The set $\Bbb N$ of positive integers is well-ordered.
\end{defn}
\section{Basic Results}
\begin{thm}\label{well-1}
For each integer $m$, the set
\[S=\{i\in\Bbb Z: i\geq m\}\]
is well-ordered.
\end{thm}
\begin{proof}
We need only show that every nonempty subset of $S$ has a least element. So let $T$ be a nonempty subset of $S$. If $T$ is a subset of $\Bbb N$, then, by \textbf{the Well-Ordering Principle}, $T$ has a least element. Hence we may assume that $T$ is not a subset of $\Bbb N$. Thus $T-\Bbb N$ is a finite nonempty set and so contains a least element $t$. Since $t\leq 0$, it follows that $t\leq x$ for all $x\in T$; so $t$ is a least element of $T$.
\end{proof}

\begin{thm}[The Division Algorithm]
Let $a$ be any integer and $b$ a positive integer. Then there exist unique integers $q$ and $r$ such that                                                                      \[a=qb+r\qquad  \mbox{where } 0\leq r<b\]
\end{thm}
\begin{proof}
The proof consists of two parts. First, we must establish the existence of the integers $q$ and $r$, and then we must show they are indeed unique.
\begin{enumerate}
  \item EXISTENCE 
  
  \noindent Consider the set $S=\{a-bn|(n\in \Bbb Z) \mbox{ and } (a-bn\geq 0)\}$. Clearly, $S\subset \Bbb W$. We shall show that $S$ contains a least element. To this end, first we will show that $S$ is a non empty subset of $\Bbb W$:
  \begin{description}
    \item[Case 1:] Suppose $a\ge 0$. Then $a=a-b\cdot 0\in S$, so $S$ contains an element.
    \item[Case 2:] Suppose $a<0$. Since $b\in \Bbb Z^+, b\geq 1$. Then $-ba\geq -a$; that is, $a-ba\geq 0$.
  \end{description}
  Consequently, $a-ba\in S$. In both cases, $S$ contains at least one element, so $S$ is a nonempty subset of $\Bbb W$. Therefore, by theorem (\ref{well-1}), $S$ contains a least element $r$. Since $r\in S$, an integer $q$ exists such that $r=a-bq$, where $r\ge 0$.
  
  \noindent To show that  $r<b$: We will prove this by contradiction. Assume $r\geq b$. Then $r-b\geq 0$. But $r-b=(a-bq)-b=a-b(q+1)$. Since $a-b(q+1)$ is of the form $a-bn$ and is greater than $0$, $a-b(q+1)\in S$; that is, $r-b\in S$. Since $b>0$, $r-b<r$. Thus, $r-b$ is smaller than $r$ and is in $S$. This contradicts our choice of $r$, so $r<b$. Thus, there are integers $q$ and $r$ such that $a=bq+r$, where $0\leq r<b$.
  
  \item UNIQUENESS 
  
  \noindent We would like to show that the integers $q$ and $r$ are unique. Assume there are integers $q,q',r$, and $r'$ such that $a=bq+r$ and $a=bq'+r'$, where $0\leq r<b$ and $0\leq r'<b$. 
  
  \noindent Assume, for convenience, that $q\geq q'$. Then $r-r'=b(q-q')$. Because $q\geq q'$, $q-q'\geq 0$ and hence $r-r'\geq 0$. But, because $r<b$ and $r'<b$, $r-r'<b$. Suppose $q>q'$; that is, $q-q'\geq 1$. Then $b(q-q')\geq b$; that is, $r-r'\geq b$. This is a contradiction because $r-r'<b$. Therefore, $q\ngtr q'$; thus, $q=q'$, and hence, $r=r'$. Thus, the integers q and r are unique, completing the uniqueness proof.
\end{enumerate}
\end{proof}
\newpage
 \begin{thebibliography}{9}

\bibitem{amsshort}
Euclid's Element:
The Thirteen Book of Euclid translated by Sir Thomas L. Heath
Cambridge University press. 1968.
\bibitem{May}
[Tom Apostle]
An Introduction to Analytic Number Theory
California Institute of Technology. 1976.

\bibitem{notsoshort}
[Jeffry Stopple]
A Primer of Analytic Number Theory From Pythagoras to Riemann
Cambridge  University  Press. 2003.

\bibitem{notsoshort}
[William Dunham]
Euler: The Master of us all.
Mathematical Association of America. 1999.
\end{thebibliography}

\end{document}
